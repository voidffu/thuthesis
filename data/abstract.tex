% !TeX root = ../thuthesis-example.tex

% 中英文摘要和关键字

\begin{abstract}
  % 手势是人类交流信息和表达意图的重要方法\cite{guo2021human}。近年来,手势驱动的人机交互技术取得了显著的进展\cite{伍杰2019基于视觉的实时手势识别方法研究, desai2017human,strickland2013using},使得手势识别(HGR)与手势生成(HGG)成为研究的热点之一。
  手语作为听障人士的重要交流媒介,其学习与普及对促进社会包容具有重要意义。然而,当前标准手语的推广面临着学习难度较高、优质教学资源匮乏,缺乏有效的学习评估机制等挑战。近年来,手势驱动的人工智能技术取得了显著的进展\cite{伍杰2019基于视觉的实时手势识别方法研究, desai2017human,strickland2013using},为解决手语辅助教学问题开辟了新可能。但目前此类系统仍面临两大挑战:(1) 手语动作评估的准确性与实时性难以保证;(2) 高质量的手语标准示范资源不足。
  针对上述问题,本文基于深度学习技术,创新性地构建了多模态手势识别与协同手势生成算法框架,旨在解决手语学习过程中的动作评估与资源匮乏问题。在此基础上,设计并实现了一套交互式手语学习助手系统。本文的主要贡献如下:

  1. 多模态手势识别算法创新:提出了一种可插拔的多策略解耦与语义集成网络(MDSI),通过“姿势-运动”与“时空-通道”特征解耦,有效降低RGB-D手势识别中的信息冗余,并结合语义滤波与标签平滑机制提升语义区分能力。实验结果表明,该方法在IsoGD和THU-READ数据集上分别超越现有最优方法2.48\%和4.33\%,同时保持轻量化设计,其附加参数量仅占主干网络的6.84\%。
  
  2. 协同手势生成算法创新:提出了 CoordSpeaker,一种新颖的字幕赋能的协同手势生成方法,实现了手势运动生成中的节奏同步和语义对齐。通过首次引入手势描述生成模块,有效解决了手势数据缺乏文本标注的问题。结合统一的运动表示与可控潜在扩散模型,实现语义与节奏的精准协同控制。实验表明,该方法可生成高质量的(Jerk $0.179\rightarrow$)、语音同步(BC $0.057\uparrow$)、语义相关(MM-Dist: $6.814$)的协同手势运动,显著领先同类方法。同时,将平均推理时间(AITS)降低至0.842秒,较现有方法加速6倍以上,极大提升了实际应用价值。
  
  3. 交互式手语学习系统构建:基于上述算法,设计并实现了一套集成实时识别、标准动作生成与交互反馈的手语学习系统。系统采用模块化架构,支持手语学习、练习评估和自由练习三种核心交互模式,并提供清晰的界面布局与友好的交互设计。在RGB-D相机上的实验验证表明,系统在12类数据上的识别准确率超过99\%,推理延迟小于0.1秒;用户研究表明,生成的手部动作自然度偏好较同类方法提升4.65\%,充分验证了所提算法的实用价值。
  
  本文提出的手势识别与生成算法不仅显著提升了识别准确率与生成质量,同时也为解决手语教学资源匮乏问题、提升手语学习效率提供了新的技术路径。

  % 关键词用“英文逗号”分隔,输出时会自动处理为正确的分隔符
  \thusetup{
    keywords = {手势识别, 手势生成, 多模态, 手语学习, 交互式系统},
  }
\end{abstract}

\begin{abstract*}
Sign language serves as a vital communication medium for the hearing-impaired population. However, due to limited learning resources and high learning costs, significant communication barriers exist between hearing-impaired and hearing individuals. While recent advances in artificial intelligence technologies have opened new possibilities for addressing this issue, sign language assisted teaching faces two critical challenges: (1) difficulties in gesture evaluation due to the lack of effective assessment mechanisms, and (2) scarcity of standard demonstration resources for learners' reference.

To address these challenges, this thesis proposes novel deep learning-based algorithms for multimodal gesture recognition and generation, and implements an interactive sign language learning assistant system. The main contributions are threefold:

1. For gesture recognition, we propose a plug-and-play Multi-strategy Decoupling and Semantic Integration (MDSI) network. Through "pose-motion" and "spatiotemporal-channel" feature decoupling, combined with semantic filtering and label smoothing mechanisms, MDSI effectively reduces information redundancy in RGB-D gesture recognition while enhancing semantic discrimination. The method achieves state-of-the-art performance on IsoGD and THU-READ datasets, surpassing existing methods by 2.48\% and 4.33\% respectively, while maintaining a lightweight design with only 6.84\% additional parameters.

2. For gesture generation, we introduce CoordSpeaker, a novel coordinated gesture generation framework driven by textual descriptions. The framework innovatively addresses the lack of descriptive text annotations in gesture data by incorporating a gesture description generation module. Through unified motion representation and controllable latent diffusion modeling, it achieves precise coordination of semantics and rhythm. Extensive experiments demonstrate that our method generates high-quality (Jerk $0.179\rightarrow$), speech-synchronized (BC $0.057\uparrow$), and semantically relevant (MM-Dist: $6.814$) coordinated gestures, outperforming existing approaches. Moreover, the optimized inference process reduces average inference time to 0.842 seconds, achieving a 6-fold speedup over existing methods.

3. Building upon these algorithms, we design and implement an interactive sign language learning assistant system. The system adopts a modular architecture integrating real-time recognition, standard gesture generation, and interactive feedback. Experiments using commercial RGB-D cameras demonstrate efficient and reliable recognition performance (>99\% accuracy for 12 gesture classes, inference time <0.1s) and high-quality gesture generation (4.65\% higher naturalness preference than comparable methods). The system implementation validates the practical value of our proposed algorithms and provides a new technical approach to addressing the scarcity of sign language teaching resources.

  % Use comma as separator when inputting
  \thusetup{
    keywords* = {Gesture Recognition, Gesture Generation, Multimodal Learning, Sign Language Learning, Interactive System},
  }
\end{abstract*}
