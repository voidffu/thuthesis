% !TeX root = ../thuthesis-example.tex

\chapter{总结与展望}
\label{sec:summary}

\section{工作总结}
本文针对手语辅助教学中存在的动作评估困难、标准教学资源匮乏等问题,基于深度学习技术开展了手势识别与生成算法研究,并设计实现了一个交互式手语学习助手系统。主要研究工作及其创新成果如下:

(1)在手势识别方面,本文提出了一种多模态的多策略解耦和语义集成网络(MDSI)。该方法通过“姿势-运动”和“时空-通道”特征解耦,有效缓解了RGB-D手势识别中的信息冗余问题。同时引入语义滤波器和标签平滑机制,增强了语义理解能力,实现了对视觉相似手势的精确区分。在IsoGD和THU-READ两个主流基准测试中,MDSI实现了最先进的识别准确率,相比现有最优方法分别提升了2.48\%和1.27\%。此外,通过可插拔设计,模型在保持性能的同时,参数量仅占主干网络的6.84\%。

(2)在手势生成方面,本文设计了一种基于描述驱动的协同手势生成框架(CoordSpeaker)。该框架首次引入手势描述生成模块,创新性地解决了手势数据缺乏描述性文本标注的问题。通过统一的运动表示方法和可控的潜在扩散模型,实现了语义和节奏的协同精确控制。大量定量与定性实验结果表明,该方法能够有效生成高质量的(Jerk $0.529\rightarrow$)、语音同步(BC $0.180\uparrow$)、语义相关(MM-Dist: $6.814$)的协同手势运动,优于现有方法。此外,通过优化的潜在扩散过程,模型的平均推理时间(AITS)仅为0.842秒,较现有方法提升了6倍以上,显著增强了系统的实用性。
% 定量结果也表明,该方法在多个评估指标上均优于现有方法:语义相关性指标MM-Dist降低至6.814,动作自然度指标FID降低了25.6\%,同时保持了良好的语音同步性(BC: 1.327)和动作多样性(L1Div: 10.861)。特别地,通过高效的分层去噪架构和

(3)基于上述算法创新,本文设计并实现了一个交互式手语学习助手系统。该系统采用模块化架构设计,集成了实时手语识别、标准动作生成和交互反馈等功能模块。系统支持手语学习、练习评估和自由练习三种主要交互模式,并提供了清晰的分区布局与友好的交互界面引导。基于商业级RGB-D相机开展的实验证明,系统能够实现高效、可靠的识别评估(12类数据识别准确率\textgreater 99\%,推理时间\textless 0.1s),并能生成高质量的、用户满意的手部动作(自然度偏好高于同类方法 4.0\%)。
系统的实现验证了本文提出算法的实用价值,为解决手语教学资源匮乏的问题提供了新的技术途径。

本文的主要创新点体现在算法研究、应用实践两个层面。首先,在算法层面,针对手势识别中的“信息冗余”与“信息缺失”挑战提出了“多策略解耦与语义集成手势识别(MDSI)”网络,提升了手势识别的效率与准确性;针对手势生成中的描述注释缺失和多模态协同控制困难的挑战,设计了“基于描述驱动的协同手势生成框架(CoordSpeaker)”,实现了高效与高质量的协同手势生成。其次,在应用层面,本文首次实现了“基于深度学习的交互式手语学习系统”,针对手语教学中的实时评估和资源匮乏挑战,提出了有效的算法解决方案,为听障人士的手语学习提供了新途径。

\section{研究展望}
本文虽在手势识别与生成领域取得了一些进展,但仍有多个方向值得进一步探索。后续研究可从以下几个维度深入:

(1)手势识别算法方面,目前采用的多分支特征融合策略仍较为简单,主要依赖于分数融合方法。未来可以探索更复杂和精细的多模态联合建模机制,如注意力引导的特征交互、跨模态对比学习等方法,以充分挖掘不同模态信息之间的互补性,进一步提升识别系统的性能和鲁棒性;
此外,针对手语识别引入更丰富的模态特征(如语音、面部表情等)是一个具有前景的研究方向。这些额外的模态信息有望为手语理解提供更丰富的语义线索,但同时也需要构建包含多模态标注的大规模数据集作为支撑;
最后,本文证明了 MDSI 灵活、可插拔的特性使其能够无缝集成到各种架构中,并能轻松适应其他领域,例如动作识别和手语识别。由于集成所需的调整很少,进一步探索 MDSI 与各种主干的集成仍然是一个有价值的方向,例如与新兴的多模态大模型结合可能带来增强的效果,并且轻微的定制可以产生有益的结果。

(2)手势生成算法方面,虽然本文提出的手势描述生成模块在一定程度上缓解了手势生成中的语义控制问题,但模型对精细手部动作(如手指细节)的描述和生成能力仍有待提升。未来可以探索基于人体骨架的分层次生成策略,将全身动作、手臂运动和手指姿态分别建模,并设计合理的协同机制,以提升生成结果的精确度和自然度。同时,引入更丰富的上下文信息和情感特征,有望进一步增强生成手势的表现力。

(3)手语学习系统方面,当前系统仅支持单个词汇级的识别和生成,在处理复杂手语句子方面仍有较大局限性。未来研究可以围绕手语语法规则建模、上下文语义理解等关键技术展开深入探索,结合自然语言建模技术,逐步实现对连续手语句子的准确识别和流畅生成;此外,系统的交互体验和个性化学习支持也有待进一步优化。当前系统交互以视觉为主,未来可以引入语音、触觉等多种交互方式,并探索基于增强现实的沉浸式学习环境;最后,系统的动态扩展能力可进一步加强,虽然目前手势生成模块支持自由灵活的词汇生成、但手势识别模块目前仅支持预定义的词汇评估。未来可以探索基于主动学习的在线增量学习机制,使系统能够从用户交互中持续学习新的手语词汇和动作,动态扩展识别和生成能力。
% 如引入多模态的交互、自适应学习策略、更丰富的学习模式等,为用户提供更加智能和沉浸式的学习环境。
