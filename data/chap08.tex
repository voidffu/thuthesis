% !TeX root = ../thuthesis-example.tex

\chapter{局限性和未来工作}
\label{sec:limitations}
\section{算法改进}
\subsection{计算复杂性}
虽然与基线模型相比,可插拔 MDSI 框架引入的额外开销极小(约 2.60M,参见第~\ref{sec:params}节),但对于实时应用而言仍然存在限制。
(1) 计算负载在很大程度上受所选主干的影响(高达约 94\%,参见第~\ref{sec:params}节),这使得较重的模型不太适合移动部署。未来的工作可以探索更轻的主干,以更好地平衡准确性和效率。 (2) MDSI 目前依赖于额外的预处理步骤,例如利用预先训练的姿势估计器和 CLIP 编码器。虽然这些步骤不会增加训练或推理期间的计算成本,但它们可能会限制 MDSI 对在线应用的适应性。简化这些流程对于实时部署至关重要。

\subsection{可扩展性和泛化性}
虽然本文的实验是在两个大型 RGB-D 动态手势数据集上进行的,但本文预计 MDSI 能够稳健地扩展到更大、更多样化的数据集。
可插拔的模块化设计确保随着数据集大小的增加,计算复杂性仍然可控,并允许根据数据集的规模和特征灵活地选择主干,从而确保一致的性能。
未来的工作可以探索更专业的手势数据集,例如专注于无接触驾驶控制的 NvGesture。

此外,MDSI 灵活、可插拔的特性使其能够轻松适应其他领域,例如动作识别和手语识别,只需进行最少的调整。
对于动作识别,可以扩展 PMD 以分离其他身体部位,而手语识别可以利用 SIN 强大的自然语言处理能力,而无需进行额外的架构更改。本文计划在未来的工作中探索这些方向。

\subsection{集成到各种架构中}
在本研究中,本文证明了 MDSI 可以无缝集成到视频手势识别中的两个主要架构中:3DCNN 和 Video Transformer(第~\ref{sec:encoder_backbone}节),同时还验证了其稳健性。
由于集成所需的调整很少,进一步探索 MDSI 与各种主干的集成仍然是一个有价值的方向,并且轻微的定制可以产生有益的结果。
例如,确定 STCD 模块的最佳插入点可能因特定的主干架构而异,并且建议根据视觉特征形状调整 SF 模块中的卷积核维度 \(d_k\)(第~\ref{sec:implementation}节)。

% \subsection{小结}
% 在本研究中,本文提出了一个可插入式框架,称为多策略解耦和语义集成手势识别网络 (MDSI),旨在解决 RGB-D 手势识别中的信息冗余 (IR) 和信息缺失 (IA) 的挑战。
% 首先,本文引入了多策略解耦网络 (MDN),通过解耦“姿势-运动”和“空间-时间-通道”特征来减轻冗余信息。
% 随后,本文提出了语义集成网络 (SIN),它通过语义过滤和标签平滑增强了语义理解,有效地指导了视觉相似手势的区分。
% MDSI 的可插入性有助于以最小的计算开销无缝集成到各种视频编码器架构中,展示了其扩展到相关领域的潜力,例如动作识别和手语识别。本文进行了广泛的实验,表明 MDSI 在两个被广泛认可的基准上超越了之前最先进的方法。
% 本文期待 MDSI 的进一步改进和优化的预处理能够增强其实时部署。


\section{应用部署与改进}
\subsection{实际部署与验证}
目前,本文已基于机器人平台所采集的手势数据完成了离线测试,验证了手势识别算法的准确性和鲁棒性。然而,系统在机器人平台上的实际控制部署尚未完成。未来工作将集中于将手势识别系统与机器人导航平台深度集成,测试其在真实场景中的表现,特别是在动态环境和复杂任务中对系统稳定性和可靠性的要求。同时,通过系统的部署和验证,进一步评估其实际应用价值,为后续改进提供参考。
\subsection{连续手势分割}
目前的手势识别系统聚焦于孤立手势,在面对实践中处理连续手势的需要时,仍存在一定的挑战。未来工作可以通过引入手势分割算法,将本研究拓展到连续手势识别领域,通过设置开始与结束手势等方式,进一步提高对连续手势的检测精度,从而提升应用的实用潜力。
\subsection{在线、实时控制}
虽然本文的机器人导航控制系统能够通过手势识别进行有效的导航控制,但在实时性和响应速度上仍有提升空间。未来的研究将集中于优化在线、实时控制能力,减少系统延迟,并进一步增强系统对动态环境变化的适应能力。通过采用更高效的计算框架和硬件加速技术,预计能够实现更快速、更稳定的在线控制体验,尤其是在复杂或紧急场景下,保证系统能迅速作出反应。
% \subsection{完善应用原型,提升用户体验}
% 虽然本研究所提出的应用原型具备基本的导航控制功能,但在用户体验、系统易用性等方面仍存在改进空间。本文计划在未来的工作中进一步优化用户界面,简化操作流程,并改进系统反馈机制,考虑采用更加直观的视觉和语音反馈,以增强用户的交互体验和满意度。

