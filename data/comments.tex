% !TeX root = ../thuthesis-example.tex

\begin{comments}
% \begin{comments}[name = {指导小组评语}]
% \begin{comments}[name = {Comments from Thesis Supervisor}]
% \begin{comments}[name = {Comments from Thesis Supervision Committee}]

  % 论文提出了……
论文研究基于 RGB-D 数据的多模态手势识别算法和手势生成算法,选题具有理论和应用价值。

论文的主要研究工作和贡献包括:1)提出了一种可插拔的多策略解耦与语义集成网络模型,通过“姿势-运动”与“时空-通道”特征解耦,提高手势识别效果;2)提出了一种结合运动表示和可控隐扩散模型的手势协同生成算法;3)构建了一套手势识别与手势生成协同框架,和一套集成了实时手势识别和标准手势生成的手语学习系统;并在公开图像数据库上进行了有效性验证。

该生政治表现良好,其研究工作表明,作者在本专业领域具有扎实的理论知识和系统技术,具有从事相关科学研究和实践创新的能力,并发表了 JCR-1 区高水平学术论文,并投稿 CCF-A 会议论文 1 篇,论文达到了硕士学位论文的研究水平,同意安排答辩,并推荐参加优秀学位论文评选。
\end{comments}
