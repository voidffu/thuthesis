% !TeX root = ../thuthesis-example.tex

\begin{denotation}[3cm]
  % 计算机视觉与深度学习相关
  \item[HGR] 手势识别(Hand Gesture Recognition)
  \item[HGG] 手势生成(Hand Gesture Generation)
  \item[IR] 信息冗余(Information Redundancy)
  \item[IA] 信息缺失(Information Absence)
  % \item[SVM] 支持向量机(Support Vector Machine)
  % \item[KNN] 最近邻算法(K-Nearest Neighbors)
  \item[CNN] 卷积神经网络(Convolutional Neural Network)
  % \item[2DCNN] 二维卷积神经网络(2D Convolutional Neural Network)
  % \item[3DCNN] 三维卷积神经网络(3D Convolutional Neural Network)
  \item[RNN] 循环神经网络(Recurrent Neural Network)
  \item[LSTM] 长短期记忆网络(Long Short-Term Memory)
  \item[ConvLSTM] 卷积长短期记忆网络(Convolutional Long Short-Term Memory)
  \item[ViT] 视觉Transformer(Vision Transformer)
  \item[VAE] 变分自编码器(Variational Autoencoder)
  % \item[GAN] 生成对抗网络(Generative Adversarial Network)
  % 本文算法相关
  \item[MDSI] 多策略解耦与语义集成网络(Multi-strategy Decoupling with Semantic Integration Network)
  \item[MDN] 多策略解耦网络(Multi-strategy Decoupling Network)
  \item[PMD] 姿态-运动解耦模块(Pose-Motion Decoupling)
  \item[STCD] 时间-空间-通道解耦模块(Spatial-Temporal-Channel Decoupling)
  \item[SIN] 语义集成网络(Semantic Integration Network)
  \item[SF] 语义滤波器(Semantic Filter)
  \item[SLS] 语义标签平滑(Semantic Label Smoothing)
  \item[CoordSpeaker] 协同手势生成算法(Coordinated Speaker)
  \item[SMPL] 蒙皮多人线性模型(Skinned Multi-Person Linear Model)
  \item[MotionLLM] 运动大语言模型(Motion Large Language Model)
  \item[LDM] 潜在扩散模型(Latent Diffusion Model)
  
  % 数据集相关
  % \item[IsoGD] 孤立手势识别数据集(IsoGD Dataset)
  % \item[THU-READ] THU-READ数据集(THU-READ Dataset)
  % \item[BEAT] 
  % \item[HumanML3D] 
  
  % 评估指标相关
  \item[Acc] 准确率(Accuracy)
  \item[FID] FID分数(Fréchet Inception Distance)
  % \item[LPIPS] LPIPS分数(Learned Perceptual Image Patch Similarity)
  \item[BC] 节拍一致性(Beat Consistency)
  \item[DIV] Diversity分数(Diversity)
  \item[AITS] 每句平均推理时间(Average Inference Time per Sentence)
  \item[MM Dist] 多模态距离(Multi-modal Distance)
  
  % 其他技术术语
  \item[RGB] RGB颜色空间(Red, Green, Blue)
  \item[RGB-D] RGB深度图像(RGB-Depth)
  % \item[IoU] 交并比(Intersection over Union)
  % \item[SGD] 随机梯度下降(Stochastic Gradient Descent)
  \item[Adam] Adam优化器(Adaptive Moment Estimation)
  \item[ReLU] ReLU激活函数(Rectified Linear Unit)
  \item[GeLU] GeLU激活函数(Gaussian Error Linear Unit)
  \item[MLP] 多层感知机(Multi-Layer Perceptron)
  \item[MSE] 均方误差(Mean Squared Error)
  \item[KL] KL散度(Kullback-Leibler)
\end{denotation}



% 也可以使用 nomencl 宏包,需要在导言区
% \usepackage{nomencl}
% \makenomenclature

% 在这里输出符号说明
% \printnomenclature[3cm]

% 在正文中的任意为都可以标题
% \nomenclature{PI}{聚酰亚胺}
% \nomenclature{MPI}{聚酰亚胺模型化合物,N-苯基邻苯酰亚胺}
% \nomenclature{PBI}{聚苯并咪唑}
% \nomenclature{MPBI}{聚苯并咪唑模型化合物,N-苯基苯并咪唑}
% \nomenclature{PY}{聚吡咙}
% \nomenclature{PMDA-BDA}{均苯四酸二酐与联苯四胺合成的聚吡咙薄膜}
% \nomenclature{MPY}{聚吡咙模型化合物}
% \nomenclature{As-PPT}{聚苯基不对称三嗪}
% \nomenclature{MAsPPT}{聚苯基不对称三嗪单模型化合物,3,5,6-三苯基-1,2,4-三嗪}
% \nomenclature{DMAsPPT}{聚苯基不对称三嗪双模型化合物(水解实验模型化合物)}
% \nomenclature{S-PPT}{聚苯基对称三嗪}
% \nomenclature{MSPPT}{聚苯基对称三嗪模型化合物,2,4,6-三苯基-1,3,5-三嗪}
% \nomenclature{PPQ}{聚苯基喹噁啉}
% \nomenclature{MPPQ}{聚苯基喹噁啉模型化合物,3,4-二苯基苯并二嗪}
% \nomenclature{HMPI}{聚酰亚胺模型化合物的质子化产物}
% \nomenclature{HMPY}{聚吡咙模型化合物的质子化产物}
% \nomenclature{HMPBI}{聚苯并咪唑模型化合物的质子化产物}
% \nomenclature{HMAsPPT}{聚苯基不对称三嗪模型化合物的质子化产物}
% \nomenclature{HMSPPT}{聚苯基对称三嗪模型化合物的质子化产物}
% \nomenclature{HMPPQ}{聚苯基喹噁啉模型化合物的质子化产物}
% \nomenclature{PDT}{热分解温度}
% \nomenclature{HPLC}{高效液相色谱(High Performance Liquid Chromatography)}
% \nomenclature{HPCE}{高效毛细管电泳色谱(High Performance Capillary lectrophoresis)}
% \nomenclature{LC-MS}{液相色谱-质谱联用(Liquid chromatography-Mass Spectrum)}
% \nomenclature{TIC}{总离子浓度(Total Ion Content)}
% \nomenclature{\textit{ab initio}}{基于第一原理的量子化学计算方法,常称从头算法}
% \nomenclature{DFT}{密度泛函理论(Density Functional Theory)}
% \nomenclature{$E_a$}{化学反应的活化能(Activation Energy)}
% \nomenclature{ZPE}{零点振动能(Zero Vibration Energy)}
% \nomenclature{PES}{势能面(Potential Energy Surface)}
% \nomenclature{TS}{过渡态(Transition State)}
% \nomenclature{TST}{过渡态理论(Transition State Theory)}
% \nomenclature{$\increment G^\neq$}{活化自由能(Activation Free Energy)}
% \nomenclature{$\kappa$}{传输系数(Transmission Coefficient)}
% \nomenclature{IRC}{内禀反应坐标(Intrinsic Reaction Coordinates)}
% \nomenclature{$\nu_i$}{虚频(Imaginary Frequency)}
% \nomenclature{ONIOM}{分层算法(Our own N-layered Integrated molecular Orbital and molecular Mechanics)}
% \nomenclature{SCF}{自洽场(Self-Consistent Field)}
% \nomenclature{SCRF}{自洽反应场(Self-Consistent Reaction Field)}
