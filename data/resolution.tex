% !TeX root = ../thuthesis-example.tex

\begin{resolution}

  % 论文提出了……

  % 论文取得的主要创新性成果包括:

  % 1. ……

  % 2. ……

  % 3. ……

  % 论文工作表明作者在×××××具有×××××知识,具有××××能力,论文××××,答辩××××。

  % 答辩委员会表决,(×票/一致)同意通过论文答辩,并建议授予×××(姓名)×××(门类)学博士/硕士学位。

  论文研究多模态手势识别与生成算法,选题具有理论意义和实用价值。

  论文完成的主要工作为:提出了一种可插拔多策略解耦与语义集成网络,提高了手势识别精度;提出了一种结合手势描述和可控隐扩散模型的手势协同生成算法,提高了生成质量与效率;构建了一套融合识别和生成的手语学习系统,促进了自主手语学习。

  论文对国内外有关文献进行了调研,对研究课题的理解和掌握达到要求,论文工作由答辩人在导师指导下独立完成。论文写作符合规范,答辩叙述清楚。
  
  论文工作表明作者在“互联网+创新设计”交叉学科领域已掌握相关基础理论和系统的专门知识,具有独立从事专门技术工作的能力。

  经答辩委员会5人无记名投票表决,一致同意通过论文答辩并建议授予房丰仪同学电子信息硕士学位。
  
  推荐该论文参加优秀学位论文评选。
  
\end{resolution}
