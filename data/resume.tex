% !TeX root = ../thuthesis-example.tex

\begin{resume}

  \section*{个人简历}

  2000 年 08 月 29 日出生于山东省济南市。

  2018 年 9 月考入厦门大学信息学院软件工程系数字媒体技术专业,2022 年 6 月以优秀毕业生身份毕业,获得工学学士学位。

  2022 年 9 月免试进入清华大学深圳国际研究生院Open Fiesta攻读电子信息工程硕士至今。


  \section*{在学期间完成的相关学术成果}

  \subsection*{学术论文}

  \begin{achievements}
    \item Fang F, Liao Z, Kan Z, Yang W, et al. MDSI: Pluggable Multi-strategy Decoupling with Semantic Integration for RGB-D Gesture Recognition [J]. 
    Pattern Recognition. (Minor Reivision)
    \item Fang F, Yang S, Yang W. CoordSpeaker: Exploiting Gesture Captioning for Coordinated Caption-Empowered Co-Speech Gesture Generation [C]. In International Conference on Computer Vision (ICCV). (Under Review)
  \end{achievements}


  % \subsection*{专利}

  % \begin{achievements}
  %   \item 任天令, 杨轶, 朱一平, 等. 硅基铁电微声学传感器畴极化区域控制和电极连接的方法: 中国, CN1602118A[P]. 2005-03-30.
  %   \item Ren T L, Yang Y, Zhu Y P, et al. Piezoelectric micro acoustic sensor based on ferroelectric materials: USA, No.11/215, 102[P]. (美国发明专利申请号.)
  % \end{achievements}

\end{resume}
