% !TeX root = ../FFY_graduate.tex

\chapter{\textcolor{red}{补充内容}}



% \section{图表示例}

% \subsection{图}

% 附录中的图片示例(图~\ref{fig:appendix-figure})。

% \begin{figure}
%   \centering
%   \includegraphics[width=0.6\linewidth]{example-image-a.pdf}
%   \caption{附录中的图片示例}
%   \label{fig:appendix-figure}
% \end{figure}


% \subsection{表格}

% 附录中的表格示例(表~\ref{tab:appendix-table})。

% \begin{table}
%   \centering
%   \caption{附录中的表格示例}
%   \begin{tabular}{ll}
%     \toprule
%     文件名          & 描述                         \\
%     \midrule
%     thuthesis.dtx   & 模板的源文件,包括文档和注释 \\
%     thuthesis.cls   & 模板文件                     \\
%     thuthesis-*.bst & BibTeX 参考文献表样式文件    \\
%     thuthesis-*.bbx & BibLaTeX 参考文献表样式文件  \\
%     thuthesis-*.cbx & BibLaTeX 引用样式文件        \\
%     \bottomrule
%   \end{tabular}
%   \label{tab:appendix-table}
% \end{table}


% \section{数学公式}

% 附录中的数学公式示例(公式\eqref{eq:appendix-equation})。
% \begin{equation}
%   \frac{1}{2 \uppi \symup{i}} \int_\gamma f = \sum_{k=1}^m n(\gamma; a_k) \mathscr{R}(f; a_k)
%   \label{eq:appendix-equation}
% \end{equation}
